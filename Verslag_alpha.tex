\documentclass[10pt,a4paper]{report}
\usepackage[utf8]{inputenc}
\usepackage{amsmath}
\usepackage{amsfonts}
\usepackage{amssymb}
\usepackage[left=2cm,right=2cm,top=2cm,bottom=2cm]{geometry}
\author{Martijn Verkleij (s1466895) and Tim Kerkhoven (s$\_\_\_\_\_\_\_$)}
\title{Report PP07}
\begin{document}


\maketitle
\tableofcontents

\chapter{Grammar} \label{Grammar}

\chapter{Structure} \label{Structure}
In this chapter the structure of the compiled code is discussed. For the structure of the language itself, see section \ref{Grammar}. 
\section{Thread branching}
Before the program can begin, the individial threads must be separated. This is done by the following set of instructions:
\begin{verbatim}
	Const 4 Reg1				// Load 4
	Compute Add SPID Reg1 Reg2	// Compute SPID + 4 in Reg2
	Jump (Ind Reg2)				// Branch out
	Jump (Abs 0xFFF)			// Main thread --> main program
	Jump (Abs 0xF0)				// Sprockell for thread pool --> thread control loop
	...
	Jump (Abs 0xFF)
\end{verbatim}

\section{Thread control loop}
Each Sprockell is redirected to a part of the program reserved for itself where it will continuously poll shared memory to see whether it has been assigned a function to execute. If this is the case, it will prepare the new thread for execution and execute the task. Whenever the task is finished, it will write the result of the function, if applicable, to its return value field, which is then marked ready. Another thread can then read this value again with a \texttt{join} statement. When a join statement is executed, that thread will then continuously poll for the return value to be ready. If it is, it is read out and its ready bit and the thread's occupation bit are unset. This means a thread with a non-null return value MUST be joined, since the thread will otherwise lock a Sprockell forever. 

A thread control loop looks as follows:
\begin{verbatim}
	Read 7 x SPID + 2		// read PC ready bit 	\newline
	Receive Reg1 \newline
	Jump (Rel 4)						// jump to end branch\newline
	Read 7 x SPID + 1		// get function address \newline
	Receive Reg2 \newline
										//Initialise AR \newline
	Jump (Ind Reg2) \newline
	Branch Reg1 (Rel -3)				// branch on non-null ready bit \newline
	Jump (Rel -7)						// jump to start
\end{verbatim}

\section{Functions}
All functions follow after the thread control loops. These are listed in order, and can be called by any thread, sequentially or in parallel.

\section{Main}
Since \texttt{main} is in fact a function itself, it looks very similar, except for the fact that it is the only function called directly, not by any PP07-code. It also differs from the others since it has an EndProg statement at the end.

\chapter{Memory model}
Sprockell has various levels of memory, namely in the form of registers, local memory and shared memory. These three levels have a specific model which is explained in this chapter.

\section{Registers}
Registers do not have a general model outside of the special registers used by the Sprockell itself. This means they can be used freely by the specific operations during compile time.

\section{Local memory}
Local memory is meant to store thread-specific data in. It will be used to store activation records. 

\section{Global memory}
Global memory has a more elaborate structure, since it stores not only data shared between threads, it also needs records to control threads' behaviour. Information about how and why these records are used can be found in section \ref{Structure}. 

\subsection{Thread control records}
Thread control records are used to control the execution of parallel threads on the number of Sprockells that are provided in the thread pool. Each Sprockell with \texttt{SPID > 0} has one of these records. Each record contains the following:

\begin{tabular}{| l || l |}
\hline
Occupation & Whether the tread is now active. This may only be flipped with a CompareAndSet. \\
\hline
PC-pointer & Used by functions to let this Sprockell execute a task in parallel. \\
\hline
PC-pointer ready & Set when the PC-pointer is ready to read. \\
\hline
Return value & Value written by the function on completion, if it has any. \\
\hline
Return value ready & Set when the thread has finished execution. \\
\hline
Argument count & Number of arguments\\
\hline
Argument pointer & pointer to arguments in shared memory \\ 
\hline
\end{tabular}

\subsection{Argument space}
Since argument size can be arbitrarily large in case of arrays (which was not implemented, however), this space is reserved for arguments for each thread. A pointer in the thread control record points here.

\subsection{Global variables}
Global variables are declared at the end of memory, to avoid collision with the argument space. Since no types are available in the language that are larger than one word, variables can be placed in one slot each.



Activation Record
\begin{itemize}
	\item Variables
	\item ARP
	\item Return value
	\item Return Address
	\item Caller's ARP
	\item Parameters
\end{itemize}


\end{document}