\documentclass[10pt,a4paper]{report}
\usepackage[utf8]{inputenc}
\usepackage{amsmath}
\usepackage{amsfonts}
\usepackage{amssymb}
\usepackage[left=2cm,right=2cm,top=2cm,bottom=2cm]{geometry}
\usepackage{wrapfig}
\usepackage{amsthm}
\usepackage{float} \floatstyle{boxed} \restylefloat{figure}
\usepackage[toc,page]{appendix}
\usepackage{listings}
\lstset{tabsize=4, numbers=left, numberstyle=\scriptsize}
\author{Martijn Verkleij (s1466895) and Tim Kerkhoven (s1375253)}
\title{PP project: PP07}
\begin{document}


\maketitle
\tableofcontents


\chapter{Features}
PP07 is a programming language written for execution on the Sprockell simulator. It was written as a Turing-complete language that supports basic arithmetic operations, if and else and while. Planned was support for functions, concurrent execution with futures and enumerations. Variable types supported are integer and boolean. A more elaborate list of all supported operations can be found in chapter \ref{chp:langdesc}. The language is case-insensitive on syntax, names of functions are however case-sensitive. 

\chapter{Problems and solutions}
\section{Checker versus Generator}
During the project it was found out that delivering all the features that were defined in the language was difficult to achieve. Later on restrictions in time forced extra features like functions and enumerations out of the window. This means that the checker phase comprises features that the generator cannot transcode. Tests are despite these shortcomings also written for many uncompilable features.

\section{Absence of concurrency}
The planned form of concurrency implementation consisted of parallelising functions, which one could join on, and implementing simple semaphore-like reentrant locks. Due to letting go of the functions functionality, concurrency fell off of its basis, which meant implementing it had to be done via an entirely new model like fork-join or simply splitting the different parts up. This proved, with respect to the little time left, impossible to implement without losing absolutely mandatory aspects of the project.

\section{Function Calls and Definitions}
In the checking phase, a control flow graph was constructed in the checker phase. This was done with context objects, which were connected together like seen in the practical exercises. During implementation a function call above the declaration of its associated function was unable to set its context, since it relied on the function context to be set, which was only possible if the function had been visited already. First the context setting of functions was delegated to the PrepWalker to try and counter this. This however proved to be impossible, since the objects were gone by the time they were read by the next walk. No solution was found for this problem, which led to the restriction that a given function call must occur after its declaration. This means functions cannot call each other.


\chapter{Language description} \label{chp:langdesc}

\section{Program Structure} \label{sec:structure}

\begin{quote}
	Note: Since functions were not implemented in the generator, this section contains info that was not completed, or maybe even poorly thought out. It is included for context. 
\end{quote}

In this section the structure of the compiled code is discussed. In the next sections the individual language constructs are discussed. 
\subsection{Thread branching}
Before the program can begin, the individial threads must be separated. This is done by the following set of instructions:
\begin{lstlisting}
	Const 4 Reg1				// Load 4
	Compute Add SPID Reg1 Reg2	// Compute SPID + 4 in Reg2
	Jump (Ind Reg2)				// Branch out
	Jump (Abs 0xFFF)			// Main thread -> main program
	Jump (Abs 0xF0)				// Sprockell for thread pool -> 
										thread control loop
	...
	Jump (Abs 0xFF)
\end{lstlisting}

\subsection{Thread control loop}
Each Sprockell is redirected to a part of the program reserved for itself where it will continuously poll shared memory to see whether it has been assigned a function to execute. If this is the case, it will prepare the new thread for execution and execute the task. Whenever the task is finished, it will write the result of the function, if applicable, to its return value field, which is then marked ready. Another thread can then read this value again with a \texttt{join} statement. When a join statement is executed, that thread will then continuously poll for the return value to be ready. If it is, it is read out and its ready bit and the thread's occupation bit are unset. This means a thread with a non-null return value MUST be joined, since the thread will otherwise lock a Sprockell forever. 

A thread control loop looks as follows:
\begin{lstlisting}
	Read 7 x SPID + 2					// read PC ready bit
	Receive Reg1
	Jump (Rel 4)						// jump to end branch
	Read 7 x SPID + 1					// get function address
	Receive Reg2
										//Initialise AR
	Jump (Ind Reg2)
	Branch Reg1 (Rel -3)				// branch on non-null ready bit
	Jump (Rel -7)						// jump to start
\end{lstlisting}


\subsection{Functions}
All functions follow after the thread control loops. These are listed in order, and can be called by any thread, sequentially or in parallel. The two are distinct in that in the sequential case an activation record is added to the local stack, and in the parallel case a free Sprockell is selected to do the task, from which a result can be read with a join statement.

\subsection{Main}
The main function is executed on the first Sprockell. It wil run sequentially until the end. At the end it will kill all remaining Sprockells by directing them to an EndProg, after which it will end itself. The main function does not have a special syntax, other that it always has its return type set to void and is called main in lowercaps, with no arguments.



Since \texttt{main} is in fact a function itself, it looks very similar, except for the fact that it is the only function called directly, not by any PP07-code. It also differs from the others since it has an EndProg statement at the end.

\section{Statements}
Statements are the elements which are executed in order to run the program.

\subsection{Variables}

\subsubsection*{Variable Declaration}
\begin{description}
	\item[Syntax:] 
		\textless type \textgreater ~\textless variablename \textgreater ~[= \textless expression \textgreater];
	\item[Example:] 
		int i; ~~~ 
		int i = 5;  ~~~ 
		int i = 5 * j + 6;
	\item[Sprill] Generated code for third example:
		\begin{enumerate}
			\item \emph{evaluate expression \emph{5 * j + 6} and push to stack}
			\item Pop RegA
			\item Store RegA \emph{MemAddr}
		\end{enumerate}
\end{description} 
A variable can be declared with scope-awareness. A variable is declared in memory, and its value may or may not be assigned, depending on if a value to assign is given.

\subsubsection*{Variable assignment}

\begin{description}
	\item[Syntax:] 
		\textless variablename\textgreater ~= \textless expression \textgreater;
	\item[Example:] 
		i = 5; ~~~ 
		i = 5 * j + 6;
	\item[Sprill] Generated code for first example (where \emph{i} is a global variable):
		\begin{enumerate}
			\item \emph{evaluate expression \emph{5} and push to stack}
			\item Pop RegA
			\item Write RegA \emph{MemAddr} 
		\end{enumerate}
\end{description} 
A variable can be assigned a value with a similar syntax as before. The resulting value is written to memory.

\subsection{Control flow statements}

\subsubsection*{If and else}

\begin{description}
	\item[Syntax:] 
		if (\textless expression\textgreater ) \textless block\textgreater ~ [else \textless block\textgreater ];
	\item[Example:] 
		if (true) \{ \ldots \} ~~~
		if (i \textless ~a) \{ i = a; \} else \{ a = i; \}
	\item[Sprill] Generated code for the second example:
		\begin{enumerate}
			\item \emph{evaluate expression \emph{i \textless ~a} and push to stack}
			\item Pop RegA
			\item Compute Equal RegA Zero RegA
			\item Branch RegA (Abs 7)
			\item \emph{evaluate \emph{i = a}}
			\item Jump (Abs 9)
			\item Nop
			\item \emph{evaluate \emph{a = i}}
			\item Nop
		\end{enumerate}
\end{description} 

The first of the basic control flow statements is the if-statement. The if-statement is extended with the else-statement. If the condition holds, the content of the if-block is executed, otherwise, either the contents of the else statement are executed, or this statement finishes. This depends on whether an else-statement is provided.


\subsubsection*{While}

\begin{description}
	\item[Syntax:] 
		while (\textless expression\textgreater ) \textless block\textgreater
	\item[Example:] 
		while (\ldots) \{\ldots\}
	\item[Sprill] Generated code for first example:
		\begin{enumerate}
			\item Jump (Abs 4)
			\item Nop
			\item \emph{evaluate \emph{\ldots}}
			\item Nop
			\item \emph{evaluate expression \emph{\ldots} and push to stack}
			\item Pop RegA
			\item Const 1 RegB
			\item Compute Equal RegA RegB RegA
			\item Branch RegA (Abs 2)
		\end{enumerate}
\end{description} 
The while statement is the second basic control flow statement. It evaluates the expression, and executes the statements in the block if the expression is true. if the statements have finished, the condition is re-evaluated, and when found to be true, the statements are once again executed.


\subsection{Block}
A block statement is a wrapper for a block. A Block opens a new scope. It is used in the control flow statements, but may also be arbitrarily used to fence off variables.

\begin{description}
	\item[Syntax:] 
		\{ \textless statement\textgreater * \}
	\item[Example:] 
		\{ int i = 5; j = 6; \}
	\item[Sprill] Generated code:
		\begin{enumerate}
			\item \emph{evaluate all statements in the block}
		\end{enumerate}
\end{description} 

\subsection{Expression}
An expression in itself is also a statement. This may be illogical, but a function call (which was planned in our language, see subsection \ref{subsec:functions}) is an expression. Interesting to note is that the resulting value has to be popped. Forgetting this results in a memory leak, as this value will otherwise never be popped.

\begin{description}
	\item[Syntax:] 
		\textless statement\textgreater
	\item[Example:] 
		\{ 2 + 3; ~~~ function(); \}
	\item[Sprill] Generated code:
		\begin{enumerate}
			\item \emph{evaluate expression}
			\item Pop Zero
		\end{enumerate}
\end{description} 

\section{Expressions}
Expressions are always part of a statement. This makes using the stack for solving expression trees possible. An expression may or may not pop a number of values from the stack as input, and return one value to the stack. 

\subsection{Plus and minus}
\begin{description}
	\item[Syntax:] \hfill \\ 
		\textless expression\textgreater ~+ \textless expression \textgreater \\
		\textless expression\textgreater ~- \textless expression \textgreater
	\item[Example:] \hfill \\
		3 + 5 + 6 + 7 
	\item[Sprill] Generated code for subtraction:
		\begin{enumerate}
			\item \emph{evaluate second expression}
			\item \emph{evaluate first expression}
			\item Pop RegA
			\item Pop RegB
			\item Compute Sub RegA RegB RegA
			\item Push RegA
		\end{enumerate}
\end{description} 
Plus and minus are only defined for expressions which return an integer. They return an integer.

\subsection{Multiplication and integer division}
\begin{description}
	\item[Syntax:] \hfill \\ 
		\textless expression\textgreater ~* \textless expression \textgreater \\
		\textless expression\textgreater ~/ \textless expression \textgreater
	\item[Example:] \hfill \\
		3 / 5 * 36
	\item[Sprill] Generated code for multiplication:
		\begin{enumerate}
			\item \emph{evaluate second expression}
			\item \emph{evaluate first expression}
			\item Pop RegA
			\item Pop RegB
			\item Compute Mul RegA RegB RegA
			\item Push RegA
		\end{enumerate}
\end{description}
Multiplication is defined for two integer expressions, and returns an integer.
Division is also defined for two integer expressions, it returns the integer division of the two expressions. 

\subsection{Exponentiation}
\begin{description}
	\item[Syntax:] \hfill \\ 
		\textless expression\textgreater \^{} \textless expression \textgreater
	\item[Example:] \hfill \\
		3 \^{} 5
	\item[Sprill] Generated code for example (this is a suboptimal solution and would have been fixed if we had more time:
		\begin{enumerate}
			\item \emph{evaluate second expression}
			\item \emph{evaluate first expression}
			\item Pop RegA
			\item Pop RegB
			\item Compute Equal RegB Zero RegD
			\item Branch RegD (Abs 15)
			\item Const 1 RegC
			\item Const 1 RegE
			\item Jump (Abs 12)
			\item Compute Mul RegA RegE RegE
			\item Compute Sub RegB RegC RegB
			\item Compute GtE RegB RegC RegD
			\item Branch RegD (Abs 10)
			\item Jump (Abs 17)
			\item Const 1 RegE
			\item Push RegE
		\end{enumerate}
\end{description}
Exponentiation is defined for two integers, and returns the result of expression 1 to the power of expression 2 as an integer.

\subsection{Boolean Operations}
\begin{description}
	\item[Syntax:] \hfill \\ 
		\textless expression\textgreater \&\& \textless expression \textgreater \\
		\textless expression\textgreater || \textless expression \textgreater
	\item[Example:] \hfill \\
		i \&\& (3 \textless ~5)
	\item[Sprill] Generated code for AND operator:
		\begin{enumerate}
			\item \emph{evaluate second expression}
			\item \emph{evaluate first expression}
			\item Pop RegA
			\item Pop RegB
			\item Compute And RegA RegB RegA
			\item Push RegA
		\end{enumerate}
\end{description}
Boolean operators require two boolean expressions as input, and deliver the boolean AND or OR result repectively as a boolean value.

\subsection{Comparison Operations}
\begin{description}
	\item[Syntax:] \hfill \\ 
		\textless expression\textgreater ~== \textless expression\textgreater \\
		\textless expression\textgreater ~\textgreater ~\textless expression\textgreater \\
		\textless expression\textgreater ~\textgreater = ~\textless expression\textgreater \\
		\textless expression\textgreater ~\textless ~\textless expression\textgreater \\
		\textless expression\textgreater ~\textless = \textless expression\textgreater \\
		\textless expression\textgreater ~!= \textless expression \textgreater
	\item[Example:] \hfill \\
		3 \textless ~5 \\
		5 != 6
	\item[Sprill] Generated code for != operator:
		\begin{enumerate}
			\item \emph{evaluate second expression}
			\item \emph{evaluate first expression}
			\item Pop RegA
			\item Pop RegB
			\item Compute NEq RegA RegB RegA
			\item Push RegA
		\end{enumerate}
		
\end{description}
Comparison operators require two boolean expressions as input, and deliver whether the two are equal, larger than, larger than or equal to, smaller than, smaller than or equal to and not equal respectively as a boolean value.

\subsection{Prefix Operations}
\begin{description}
	\item[Syntax:] \hfill \\ 
		- \textless expression\textgreater \\
		! \textless expression\textgreater
	\item[Example:] \hfill \\
		!True == False \\
		-5
	\item[Sprill] Generated code for NOT operator:
		\begin{enumerate}
			\item \emph{evaluate expression}
			\item Pop RegA
			\item Const 1 RegB
			\item Compute Xor RegA RegB RegA
			\item Push RegA 
		\end{enumerate}
\end{description}
The minus operator accepts an integer expression as input and returns the negative equivalent of the same integer as an integer.
The not operator accepts a boolean expression as input and returns the opposing value as a boolean.

\subsection{Non-operations}
\subsubsection*{Parantheses}
\subsection{Prefix Operations}
\begin{description}
	\item[Syntax:] \hfill \\ 
		( \textless expression\textgreater )
	\item[Example:] \hfill \\
		(3 + 2) * 5 
\end{description}
Not implemented in SPRiL-code, parentheses serve in the parsing phase as a way to give explicit priority in resolving expressions.

\subsubsection*{ID's, numbers and boolean values}
\begin{description}
	\item[Syntax:] \hfill \\ 
		 \textless variable name\textgreater \\
		 \textless number\textgreater \\
		 true\\
		 false
	\item[Example:] \hfill \\
		i\\ 
		5\\
		false
	\item[Sprill] Generated code for ID, where ID is global:
		\begin{enumerate}
			\item Read \emph{MemAddr}
			\item Receive RegA
			\item Push RegA
		\end{enumerate}
	\item[Sprill] Generated code for number:
		\begin{enumerate}
			\item Const \emph{number} RegA
			\item Push RegA
		\end{enumerate}
\end{description}
These serve as the leaves of all expressions inside the language. Variable names have to be resolved from memory, either local or shared, and return its value, which is of the variable's type given at declaration. Numbers represent their own value in integer form, and true and false represent their values in boolean form.

\subsection{Functions} \label{subsec:functions}
\subsubsection{Function Definition}
A function definition is a statement. It will not be executed by default, it will only be executed if the code does a parallel or sequential function call. It has a number of arguments, and has a return type, where void defines the absence of a meaningful return 
type.

\begin{description}
	\item[Syntax:] 
		\textless type\textgreater ~\textless ID\textgreater ~( [ \textless type\textgreater ~\textless argname\textgreater ~[ , \textless type\textgreater ~\textless argname\textgreater ~]* ] ) \textless block\textgreater
	\item[Example:] 
		int function(int a, int b, int c) \{ \ldots \} ~~~
		void doSomething() \{ \ldots \}
\end{description} 



\subsubsection{Sequential function call}
A function can be called in sequence. This adds a new activation record to memory. It will then set the PC and start execution. The function itself is then responsible for adding the return value to the stack and discarding the AR, after which it returns the PC to the value of the return pointer.


In a sequential function call the caller must set up some things before the function can be executed. An activation record (Section \ref{fig:actrecord}) must be made. 



\subsubsection{Parallel function call}






\section{Memory model}
Sprockell has various levels of memory, namely in the form of registers, local memory and shared memory. These three levels have a specific model which is explained in this chapter.

\subsection{Registers}
Registers do not have a general model outside of the special registers used by the Sprockell itself. This means they can be used freely by the specific operations during compile time.

\subsection{Local memory}
Local memory is currently only used to store the stack in. The stack stores expression results, which are recursively evaluated.

\subsection{Global memory}
Global memory stores all variables (since by the absence of functions they are all globally accessible).

Global memory has a more elaborate structure, since it stores not only data shared between threads, it also needs records to control threads' behaviour. Information about how and why these records are used can be found in section \ref{sec:structure}. 

\subsubsection{Thread control records}
Thread control records are used to control the execution of parallel threads on the number of Sprockells that are provided in the thread pool. Each Sprockell with \texttt{SPID > 0} has one of these records. Each record contains the following:

\begin{tabular}{| l | l |}
\hline
Occupation 
	& Whether the tread is now active. This may only be flipped with a CompareAndSet. \\
PC-pointer 
	& Used by functions to let this Sprockell execute a task in parallel. \\
PC-pointer ready 
	& Set when the PC-pointer is ready to read. \\
Return value 
	& Value written by the function on completion, if it has any. \\
Return value ready 
	& Set when the thread has finished execution. \\
Argument count 
	& Number of arguments. \\
Argument pointer 
	& pointer to argument space in shared memory. \\ 
\hline
\end{tabular}

\subsubsection{Argument space}
Since argument size can be arbitrarily large in case of arrays (which was not implemented, however), this space is reserved for arguments for each thread. A pointer in the thread control record points here.
       
\subsubsection{Global variables}
Global variables are declared at the end of memory, to avoid collision with the argument space. Since no types are available in the language that are larger than one word, variables can be placed in one slot each.


\chapter{Software description}

\section{ANTLR4 grammar}

\begin{description}
	\item[File:] /src/grammar/grammar.g4
	\item[Description:] \hfill \\
		Contains the ANTLR4 grammar for PP07.
\end{description}


\section{Java classes}

\begin{description}
	\item[File:] /src/grammar/PP07Checker.java
	\item[Description:] \hfill \\
		Comprises the checker phase of compiling PP07 code. It uses a tree listener to iterate over the code, checking input code for syntax errors, type checking for variables with a symbol table. The checking phase is done in three passes. One for function declaration (FunctionWalker), one for run checking (PrepWalker), and this one to do the rest of the work.
\end{description}

\begin{description}
	\item[File:] /src/grammar/PP07PrepWalker.java
	\item[Description:] \hfill \\
		Runs before PP07Checker's tree visitor with its own to fill in run declarations, which MUST occur after fucntion declaration checking and before join checking (see concurrency). 
\end{description}

\begin{description}
	\item[File:] /src/grammar/PP07FunctionWalker.java
	\item[Description:] \hfill \\
		Runs before PP07PrepWalker's tree visitor with its own to fill in function declarations, must occur before join checking (see concurrency) and function call checking. 
\end{description}

\begin{description}
	\item[File:] /src/grammar/PP07Generator.java
	\item[Description:] \hfill \\
		Generates SPRiL code for valid PP07 code. It uses a visitor to walk through the parse tree and generates imperative programming code for execution on a single Sprockell. The generator generates a haskell file "program.hs" in the folder /sprockell/src. This can then be executed with \texttt{ghc program.hs} from a command line.
\end{description}

\begin{description}
	\item[File:] /src/grammar/PP07Compiler.java
	\item[Description:] \hfill \\
		The wrapper class used to generate SPRiL code for execution on Sprockells. It uses PP07Checker to validate the code, and after checking generates SPRiL code. This class can be run and requests a filename from a file inside /src/sample. This code is then written to /sprockell/src. See PP07Generator for more details on running the generated code. 
\end{description}

\begin{description}
	\item[File:] /src/grammar/Functions.java
	\item[Description:] \hfill \\
		For usage in the checker phase. Models a list of functions that are checked for existence and type checked on variables.  
\end{description}

\begin{description}
	\item[File:] /src/grammar/Runs.java
	\item[Description:] \hfill \\
		For usage in the checker phase. Models a list of Runs (construct for parallel execution of a function), to check whether join statements have a Run statement associated, and to do type-checking on the return type.
\end{description}

\begin{description}
	\item[File:] /src/grammar/Locks.java
	\item[Description:] \hfill \\
		For usage in the checker phase. Models a list of Locks, to check whether locks are declared.
\end{description}

\begin{description}
	\item[File:] /src/grammar/SymbolTable.java
	\item[Description:] \hfill \\
		For usage in the checker phase. Reused from the practical excercises, this class models a symbol table. It facilitates type checking in a global scope and a local scope. 
\end{description}

\begin{description}
	\item[File:] /src/grammar/Result.java
	\item[Description:] \hfill \\
		Reused from the practical excercises. Holds the calculated control flow graph, offsets and types from the checker phase. 
\end{description}

\begin{description}
	\item[File:] /src/grammar/Type.java
	\item[Description:] \hfill \\
		Enumeration of the types used in our programming language
\end{description}

\begin{description}
	\item[File:] /src/grammar/TypeSize.java
	\item[Description:] \hfill \\
		Helper class to determine the size of a variable in memory. Useful for larger types like arrays or strings, it is not used in the generated part of the compiler.
\end{description}

\begin{description}
	\item[File:] /src/grammar/Value.java
	\item[Description:] \hfill \\
\end{description}

\begin{description}
	\item[File:] /src/grammar/Op.java
	\item[Description:] \hfill \\
\end{description}

\chapter{Test plan and results}

\section{Test plan}
Three types of testing have been performed: syntax testing, contextual testing and semantic testing. Of the three types, the former two are automated, and the latter has to be done manually.
\subsubsection*{Syntax Testing}
For the syntax testing, most, if not all, of the features in the grammar have been put in a single program. By testing on the grammar, not the implementation, we can also test for features we did not implement. The tester tests four different files, each with a different error, in addition a the correct file.
\subsubsection*{Contextual Testing}
The contextual testing is done in the exact same way as the syntax testing, only with different errors.
\subsubsection*{Semantic Testing}
The semantic tests have to be performed manually, each test will either print a character, loop indefinitely or crash. This has to be checked with ghci.

\section{Results}
All tests show the expected results, and are therefore passed.

\chapter{Conclusion}
\section{Language}
The language is very limited in what it can do, but implements the most basic of structures, making it Turing-complete (at least). We made it extendable to include functions, which would increase the power of the language, since concurrent execution would not be much of a problem any more. The language shares a lot of syntactic elements with Java and C, which makes it seem more powerful than it is. It however simplified making test programs, which was helpful. 

\section{Project}
The project was a quite stressful one. We wasted a lot of time in the few days after the language definition, which delayed implementation of functionality and made us have to sacrifice functionality. The project was challenging, we simply did not grasp the size of the project enough. Visualising all the work was hard. 
\section{Module}
We found the module quite packed and stressfull. The subjects costed a lot of time. We were, even while putting in a lot of time, quite on the slow end of the exercises, making catching up hard. This also had its effect on Concurrent Programming, where the lack of incentive left it in the dust. This was not helped by the fact that missing one of the sessions was okay, which was not communicated well. 




\begin{appendices}

\chapter{Grammar}

\begin{lstlisting}

grammar Grammar;

//@header{package grammar;}

// Program consists of one or more statements
program	: stat+;

/* Statements
* 	declStat	- Declaration and Optional Assignment
*		Declare statement, declare a (optionally global) variable and 
*		optionally assign an expression to it
*	assStat		- Assignment
*		Assign statement, assign an expression to a variable
*	enumStat	- Enumeration Declaration
*		Enumeration statement, declare an (optionally global) enumeration, 
*		and assign values to it. Assignment must occur in this phase
*	ifStat		- If
*		If statement, with expression between parentheses followed by a 
*		statement and optionally 'else' and another statement
*	whileStat	- While
*		While statement, with expression between parentheses followed by a
*		statement
*	blockStat	- Block
*		Block statement, see 'block'
*	funcStat	- Function Declaration
*		Function declaration, consisting of a type, id, parameters and a 
*		block statement
*	exprStat	- Expression
*		Expression statement, an expression can be a statement, e.g. a  
*		function call
*	runStat		- Run
*		Run statement, which creates a new thread that executes a function 
*		defined in the call to run, parameters to this function are also  
*		passed along at this stage as expressions
*	lockStat	- Lock
*		Looks up a lock in memory, if it's zero, set it to one and continue, 
*		if it's one, try again. This effectively "locks" the lock
*	unlockStat	- Unlock
*		Set a lock in memory to zero, effectively "unlocking" the lock
*	returnStat	- Return
*		Return statement, sets return value of the function. Must be located 
*		at top-level of the function scope
*/


stat	: GLOBAL? type ID (ASS expr)? SEMI						#declStat
		| ID ASS expr SEMI 										#assStat
		| ENUM ID ASS LBRACE EID (COMMA EID)* RBRACE SEMI		#enumStat
		| IF LPAR expr RPAR block (ELSE block)?					#ifStat
		| WHILE LPAR expr RPAR block 							#whileStat
		| block													#blockStat
		| type ID LPAR (type ID (COMMA type ID)*)? RPAR block	#funcStat
		| expr SEMI												#exprStat
		| RUN ID LPAR (ID (COMMA expr)*) RPAR SEMI				#runStat
		| LOCK ID SEMI											#lockStat
		| UNLOCK ID SEMI										#unlockStat
		| RETURN expr SEMI										#returnStat
		;

/*	Block Statement
*		Contains zero or more statements between two braces
*/
block 	: LBRACE stat* RBRACE
	;

/* TypeSize
*	INT
*		Integer type
*	BOOL
*		Boolean type
*	VOID
*		Empty type
*/
type	: INT													#intType
		| BOOL													#boolType
		| VOID													#voidType
		;

/* Expressions
*	funcCall 	- Function Call Expression
*		Call a function
*	joinExpr	- Join Expression
*		After new a thread is created with 'run', it's return value can be 
*		obtained using 'join'. Note that this will block the join-calling 
*		thread if the thread has not finished its execution yet. The 
*		expression will block until the return value of the thread is known.
*		It also frees up the Sprockell which ran the function, so it is
*		important that this function is always called for any threads made. 
*	lockedExpr	- Locked Expression
*		Checks if a lock is 'locked', returns a boolean. See 'lock' and 'unlock'.
*	plusExpr	- Addition Expression
*		Addition and subtraction
*	multExpr	- Multiplication Expression
*		Multiplication and integer division
*	expExpr		- Exponent Expression
*		Exponentiation b^n. Where b is any integer and n is any non-negative 
*		integer
*	boolExpr	- Boolean Expression
*		And and or operations
*	cmpExpr		- Comparison Expression
*		Comparison operations
*	prfExpr		- Prefix Expression
*		Negate and not operations
*	parExpr		- Parentheses Expression
*		Parentheses
*	idExpr		- ID Expression
*		An ID
*	numExpr		- Number Expression
*		A number
*	eidExpr		- Enumeration ID Expression
*		An enum ID
*	trueExpr	- True Expression
*		True
*	falseExpr	- False Expression
*		False
*/
expr	: ID LPAR (expr (COMMA expr)*)? RPAR					#funcCall
		| JOIN ID												#joinExpr
		| LOCKED ID												#lockedExpr
		| expr plusOp expr										#plusExpr
		| expr multOp expr										#multExpr
		| expr expOp expr										#expExpr
		| expr boolOp expr										#boolExpr
		| expr cmpOp expr										#cmpExpr
		| prfOp expr											#prfExpr
		| LPAR expr RPAR 										#parExpr
		| ID													#idExpr
		| NUM													#numExpr
		| EID													#eidExpr
		| TRUE													#trueExpr
		| FALSE													#falseExpr
		;

// Operators
plusOp	: PLUS | MINUS; 				// + and -
multOp	: STAR | SLASH;					// * and /
expOp	: CARET;						// ^
boolOp	: AND | OR;						// AND and OR
cmpOp	: EQ | GT | GE | LT | LE | NE;	// ==, >, >=, <, <= and !=
prfOp	: MINUS | NOT;					// - and !

// Keywords
INT: 	I N T;
BOOL: 	B O O L;
VOID:	V O I D;
ENUM: 	E N U M;

GLOBAL: G L O B A L;
IF: 	I F;
ELSE: 	E L S E;
WHILE: 	W H I L E;
RUN:	R U N;
JOIN:	J O I N;
LOCK:	L O C K;
UNLOCK:	U N L O C K;
LOCKED:	L O C K E D;
RETURN: R E T U R N;
TRUE:	T R U E;
FALSE:	F A L S E;

// Symbols
ASS: 	'=';

EQ: 	'==';
GT: 	'>';
GE: 	'>=';
LT:     '<';
LE:     '<=';
NE:     '!=';

AND:	'&&';
OR:		'||';
NOT:	'!';

LBRACE: '{';
RBRACE: '}';
LPAR:   '(';
RPAR:   ')';
LBRACK:	'[';
RBRACK:	']';

SEMI:   ';';
DQUOTE: '"';
SQUOTE: '\'';
COLON:  ':';
COMMA:  ',';
DOT:    '.';
AMP:	'&';
UND:	'_';

PLUS:   '+';
MINUS:  '-';
SLASH:  '/';
STAR:   '*';
CARET:	'^';

/* IDs, numbers and strings
*	ID
*		Starts with a lowercase letter, followed by zero or more letters,
*		underscores and digits
*	EID
*		Starts with an uppercase letter, followed by zero or more letters, 
*		underscores and digits
*	NUM
*		Consists of one or more digits
*	STR
*		Consists of any characters between two double quotes
*/
ID: 	LCASEL (LETTER | UND | DIGIT)*;
EID:	UCASEL (UCASEL | UND | DIGIT)*;
NUM:	DIGIT+;
STR: 	DQUOTE .*? DQUOTE;

// Fragments
fragment LETTER: [a-zA-Z]; 	// Any letter
fragment UCASEL: [A-Z];		// Any uppercase letter
fragment LCASEL: [a-z];		// Any lowercase letter
fragment DIGIT: [0-9];		// Any digit

// Below are all letters in either upper or lower case, these are used to 
// define keywords.
fragment A: [aA];
fragment B: [bB];
fragment C: [cC];
fragment D: [dD];
fragment E: [eE];
fragment F: [fF];
fragment G: [gG];
fragment H: [hH];
fragment I: [iI];
fragment J: [jJ];
fragment K: [kK];
fragment L: [lL];
fragment M: [mM];
fragment N: [nN];
fragment O: [oO];
fragment P: [pP];
fragment Q: [qQ];
fragment R: [rR];
fragment S: [sS];
fragment T: [tT];
fragment U: [uU];
fragment V: [vV];
fragment W: [wW];
fragment X: [xX];
fragment Y: [yY];
fragment Z: [zZ];

// Skip comments (line and block) and whitespace
COMMENT: (('/*'.*? '*/')|('//' .*? '\n')) -> skip;
WS: [ \t\r\n]+ -> skip;


\end{lstlisting}

\chapter{Extended test program}

\end{appendices}


\end{document}